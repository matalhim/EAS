\chapter*{Введение}
\addcontentsline{toc}{chapter}{Введение}
\label{ch:intro}

     В области энергий выше $10^{15}$ эВ интенсивность потока первичных космических лучей (ПКЛ) довольно мала, что делает затруднительным проведение прямых измерений их энергии и массового состава с помощью детекторов, размещенных на космических аппаратах или аэростатах. Поэтому единственным источником информации о свойствах первичных космических лучей в данной энергетической области на сегодняшний день являются широкие атмосферные ливни (ШАЛ), которые представляют собой ядерно-электромагнитные каскады, инициированные при взаимодействии первичных частиц с ядрами атомов воздуха.
     Уникальная научная установка «Экспериментальный комплекс НЕВОД» (ЭК НЕВОД), расположенная на территории НИЯУ МИФИ, позволяет проводить фундаментальные и прикладные исследования с использованием природных потоков частиц на поверхности Земли. 
     
	Одной из наиболее интересных задач ЭК НЕВОД является изучение групп мюонов, представляющих собой одновременное (в пределах десятков наносекунд) прохождение через установку проникающих частиц с практически параллельными треками, с помощью прецизионного трекового детектора ДЕКОР (проводится с 2002 г.) \cite{bogdanov2010investigation, kokoulin2021muon} и черенковского водного калориметра НЕВОД (с 2013 г.) \cite{yurina2021measurements}. Исследование мюонной компоненты ШАЛ в целом дает возможность получать информацию о массовом составе ПКЛ и проверять модели адронных взаимодействий при сверхвысоких энергиях. Функционирование в составе ЭК НЕВОД установки НЕВОД-ШАЛ поможет осуществить привязку событий с группами мюонов к традиционным методам регистрации ШАЛ, где определяется положение оси, мощность ливня, направление, и изучить, например, пространственное распределение мюонов.
    



\endinput