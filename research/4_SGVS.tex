\chapter*{Система глобальной временной синхронизации (СГВС)}
\addcontentsline{toc}{chapter}{Система глобальной временной синхронизации (СГВС)}
\label{ch:intro}

Для обеспечения синхронной работы кластеров установки, а также привязки регистрируемых событий к мировому времени используется система глобальной временной синхронизации (СГВС) \cite{amelchakov2022nevod}. В состав СГВС входят: модуль глобальной временной синхронизации (МГВС), антенна GPS/ГЛОНАСС и управляющая ЭВМ. 

Основными функциями модуля МГВС являются:
\begin{itemize}
    \item раздача единой тактовой частоты 100 МГц в синхронизируемые устройства;
    \item раздача временных меток в синхронизируемые устройства для синхронного запуска их локальных часов;
    \item независимая синхронизация любого из каналов;
    \item синхронизация локальных часов реального времени с внешним приемником GPS;
    \item передача по сети Ethernet времени локальных часов и GPS-приемника.
\end{itemize}

Точность синхронизации локальных часов кластеров установки составляет 10 нс (1 период тактового генератора, установленного в модуле МГВС). Система СГВС применяется и для синхронизации локальных часов триггерной системы НЕВОД-ДЕКОР, работающей на собственной тактовой частоте 40 МГц. Для этого с выхода МГВС передается только временная метка. Точность синхронизации между НЕВОД-ШАЛ и НЕВОД-ДЕКОР при этом составляет около 25 нс.




\endinput